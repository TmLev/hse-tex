\begin{frame}

\frametitle{\TeX: а зачем?}

\begin{itemize}[<+->]
    \item WYSIWYM позволяет отделить содержимое от его представления
    \item Компиляция в PDF (можно и в другие форматы)
    \item Open source: \href{https://github.com/latex3}{\color{blue} \LaTeX \enspace GitHub repository}
    \item Кросс-платформенность
    \item Автоматизация рутинных процессов:
    \begin{itemize}
        \item Нумерация списков, страниц, формул, таблиц, \dots
        \item Составление таблицы содержания
        \item Выравнивание всего и вся
        \item Генерация списка источников (BibTeX)
    \end{itemize}
    \item Расширяемость как у языков программирования, можно:
    \begin{itemize}
        \item Писать свои команды (функции)
        \item Подключать third-party расширения (библиотеки)
        \item Версионировать при помощи VCS (исходники -- просто текст)
        \item Разбивать большой проект на небольшие файлы
    \end{itemize}
\end{itemize}

\end{frame}
