\begin{frame}[fragile]{Базовые команды: begin/end}

\begin{itemize}
    \item Команды \lstinline!\begin{}! и \lstinline!\end{}! создают новый блок
    \item Такой блок называется ``окружением'' (environment)
    \item Окружением отделяют особые части документа: \pause
    \begin{itemize}[<+->]
        \item Сам документ является окружением
        \item Математическое окружение
        \item Списки (нумерованные и нет, специальные bullets, \dots)
    \end{itemize}
\end{itemize}

\begin{onlyenv}<2>
    \begin{block}{Пример}
        \begin{lstlisting}
    \begin{document}
        Hello, World!
    \end{document}
        \end{lstlisting}
    \end{block}
\end{onlyenv}

\begin{onlyenv}<3>
    \begin{block}{Пример}
        \begin{lstlisting}
    % \usepackage{amsmath}
    % \renewcommand{\b}[1]{\mathbf{#1}}
    \begin{align}
        \b{d} f(\b{X}) = -\b{X}^{-1} [\b{d X}] \, \b{X}^{-1}
    \end{align}
        \end{lstlisting}
    \end{block}
\end{onlyenv}

\begin{onlyenv}<4>
    \begin{block}{Пример}
        \begin{lstlisting}
    \begin{enumerate}
        \item First
        \item Second
    \end{enumerate}
    
    \begin{itemize}
        \item Item
        \item One more
    \end{itemize}
        \end{lstlisting}
    \end{block}
\end{onlyenv}

\end{frame}