% ------------------------------------------------------------

\begin{frame}[fragile]{Базовые команды: математика}

\begin{itemize}[<+->]
    \item Формулы внутри текста оборачиваются \$ с обеих сторон
    \item Для отдельного блока с формулами можно использовать \lstinline!\begin{align}! и \lstinline!\end{align}!
\end{itemize}

\begin{onlyenv}<1>
    \begin{block}{Пример}
        \begin{lstlisting}
    \documentclass{article}
    
    \begin{document}
        The solution for the equation $x^3 + 2 = 10$ is $x = 2$.
    \end{document}
        \end{lstlisting}
    \end{block}
\end{onlyenv}

\begin{onlyenv}<2>
    \begin{block}{Пример}
        \begin{lstlisting}
    \documentclass{article}
    
    % Import `align` environment (and more).
    \usepackage{amsmath}
    
    \begin{document}
        Solve the following equation:
        \begin{align}
            z = 3y_1 + 2y_2 + 4y_3
        \end{align}
    \end{document}
        \end{lstlisting}
    \end{block}
\end{onlyenv}

\end{frame}

% ------------------------------------------------------------

\begin{frame}[fragile]{Базовые команды: математика}

Можно ещё проще, без align:

\begin{block}{Пример}
    \begin{lstlisting}
    \[
        x + y = 45
    \]
    \end{lstlisting}
\end{block}

Либо equation для простых уравнений

\begin{block}{Пример}
    \begin{lstlisting}
    \begin{equation}
        x + y = 45
    \end{equation}
    \end{lstlisting}
\end{block}

\end{frame}

% ------------------------------------------------------------

\begin{frame}[fragile]{Базовые команды: математика}

Сила align в поколоночном выравнивании

\begin{block}{Пример}
    \begin{lstlisting}
    % \usepackage{amsmath}
    % \renewcommand{\b}[1]{\mathbf{#1}}

    \begin{align}
        \b{d} f(\b{x}) 
        &= \b{d} \log(1 + \b{x^T A x}) 
        = \b{d} \log (1 + \langle \b{A x}, \b{x} \rangle) \\
        &=\frac{\b{d} (1 + \langle \b{A x}, \b{x} \rangle)}
               {1 + \langle \b{A x}, \b{x} \rangle}
        = \frac{2 \langle \b{A x}, \b{d x} \rangle}
               {1 + \langle \b{A x}, \b{x} \rangle}
    \end{align}
    \end{lstlisting}
\end{block}

\end{frame}

% ------------------------------------------------------------

\begin{frame}[fragile]{Базовые команды: математика}

\begin{itemize}[<+->]
    \item Умножение задаётся \textbf{не звёздочкой}
    \item Для этого есть \lstinline!\cdot! и \lstinline!\times!
    \item Большинство известных функций и кванторов уже доступны
    \item Для дробей используется \lstinline!\frac{1}{2}!
    \item Обычные скобки занимают высоту одного символа
    \item Автомасштабирующиеся скобки -- \lstinline!\left(! и \lstinline!\right)!
    \item Индексация и степени требуют \{\}, если больше одного символа
\end{itemize}

\begin{onlyenv}<1-2>
    \begin{block}{Пример}
        \begin{lstlisting}
        Incorrect: $2 * 3 = 6$.
        
        Better: $2 \cdot 3 = 6$ or $2 \times 3 = 6$.
        \end{lstlisting}
    \end{block}
\end{onlyenv}

\begin{onlyenv}<3>
    \begin{block}{Пример}
        \begin{lstlisting}
        $\log x + \sin x = \cos x - \sqrt[5]{z}$
        $\sum_{i = 1}^{n} i + 2 \neq 10$
        $x \cap y = z$
        \end{lstlisting}
    \end{block}
\end{onlyenv}

\begin{onlyenv}<4>
    \begin{block}{Пример}
        \begin{lstlisting}
        $\frac{x}{y} = \frac{1}{1 + \frac{1}{z}}$
        \end{lstlisting}
    \end{block}
\end{onlyenv}

\begin{onlyenv}<5>
    \begin{block}{Пример}
        \begin{lstlisting}
        $\frac{x}{y} = (
            \frac{1}{1 + \frac{1}{z}}
            - \frac{z}{c}
        )$
        \end{lstlisting}
    \end{block}
\end{onlyenv}

\begin{onlyenv}<6>
    \begin{block}{Пример}
        \begin{lstlisting}
        $\frac{x}{y} = \left(
            \frac{1}{1 + \frac{1}{z}}
            - \frac{z}{c}
        \right)$
        \end{lstlisting}
    \end{block}
\end{onlyenv}

\begin{onlyenv}<7>
    \begin{block}{Пример}
        \begin{lstlisting}
        $x_1 + x_12 + y^2 + y^22$
        $x_1 + x_{12} + y^2 + y^{22}$
        \end{lstlisting}
    \end{block}
\end{onlyenv}

\end{frame}

% ------------------------------------------------------------

\begin{frame}[fragile]{Базовые команды: математика}

Набор пакетов, которых должно хватить для большинства выкладок

\begin{block}{Пример}
    \begin{lstlisting}
    % Math symbols.
    \usepackage{amssymb}
    
    % Theorems.
    \usepackage{amsthm}
    
    % Insert text into formulas.
    \usepackage{amstext}      
    
    % Math fonts.
    \usepackage{amsfonts}
    
    % Smart comma: $0,2$ --- number, $0, 2$ --- enumeration.
    \usepackage{icomma}
    
    % Tweaked `amsmath`.
    \usepackage{mathtools}
    \end{lstlisting}
\end{block}

\end{frame}
