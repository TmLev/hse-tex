% ------------------------------------------------------------

\begin{frame}[fragile]{Базовые команды: текст}

\begin{itemize}[<+->]
    \item Текст вводится как обычно
    \item Количество пробелов не влияет на расстояние между словами
    \item Абзацы разделяются пустой строкой
    \item Перенос строки внутри абзаца при помощи \lstinline!\\!
\end{itemize}

\begin{onlyenv}<1>
    \begin{block}{Пример}
        \begin{lstlisting}
    \documentclass{article}
    
    \begin{document}
        Hello, World!
    \end{document}
        \end{lstlisting}
    \end{block}
\end{onlyenv}

\begin{onlyenv}<2>
    \begin{block}{Пример}
        \begin{lstlisting}
    \documentclass{article}
    
    \begin{document}
        Hello,            World!
    \end{document}
        \end{lstlisting}
    \end{block}
\end{onlyenv}

\begin{onlyenv}<3>
    \begin{block}{Пример}
        \begin{lstlisting}
    \documentclass{article}
    
    \begin{document}
        Hello, World!
        
        Hello, \LaTeX!
    \end{document}
        \end{lstlisting}
    \end{block}
\end{onlyenv}

\begin{onlyenv}<4>
    \begin{block}{Пример}
        \begin{lstlisting}
    \documentclass{article}
    
    \begin{document}
        LaTeX is a high-quality typesetting system. \\
        It includes features designed for the production 
        of technical and scientific documentation. 
    \end{document}
        \end{lstlisting}
    \end{block}
\end{onlyenv}

\end{frame}

% ------------------------------------------------------------

\begin{frame}[fragile]{Базовые команды: текст}

\begin{itemize}[<+->]
    \item \lstinline!\section{Section Name}! делит текст на логические части
    \item \lstinline!\subsection{Subsection Name}! -- подсекции
\end{itemize}

\begin{onlyenv}<1>
    \begin{block}{Пример}
        \begin{lstlisting}
    \documentclass{article}
    
    \begin{document}
        \section{Introduction}
        LaTeX is the de facto standard for the communication 
        and publication of scientific documents.
        
        \section{Sponsorship}
        You can also sponsor the work of LaTeX team members 
        through the GitHub sponsor program.
    \end{document}
        \end{lstlisting}
    \end{block}
\end{onlyenv}

\begin{onlyenv}<2>
    \begin{block}{Пример}
        \begin{lstlisting}
    \documentclass{article}
    
    \begin{document}
        \section{GitHub}
        \subsection{What is it?}
        GitHub is a provider of Internet hosting for 
        software development and version control using Git.
        
        \subsection{Features}
        It offers the distributed version control and 
        source code management functionality of Git, 
        plus its own features.
    \end{document}
        \end{lstlisting}
    \end{block}
\end{onlyenv}

\end{frame}

% ------------------------------------------------------------

\begin{frame}[fragile]{Базовые команды: текст}

\begin{itemize}[<+->]
    \item \lstinline!\textbf{}! даёт жирный шрифт (bold)
    \item \lstinline!\textit{}! наклоняет текст (italic)
    \item Кавычки-ёлочки задаются как $<<$текст$>>$
    \item Кавычки сверху задаются как $``$текст$''$
\end{itemize}

\begin{onlyenv}<1-2>
    \begin{block}{Пример}
        \begin{lstlisting}
        \documentclass{article}
        
        \begin{document}
            Hello, \textbf{World}! So nice to \textit{see} you!
        \end{document}
        \end{lstlisting}
    \end{block}
\end{onlyenv}

\begin{onlyenv}<3-4>
    \begin{block}{Пример}
        \begin{lstlisting}
        \documentclass{article}
        
        \begin{document}
            <<Higher School of Economics>>
            
            ``Higher School of Economics''
        \end{document}
        \end{lstlisting}
    \end{block}
\end{onlyenv}

\end{frame}

% ------------------------------------------------------------

\begin{frame}[fragile]{Базовые команды: текст}

\begin{itemize}
    \item Указать название документа можно при помощи \lstinline!\title{}!
    \item Автор задаётся командой \lstinline!\author{}!
    \item Также можно вставить дату: \lstinline!\date{}!
    \item \lstinline!\maketitle! -- автоматическое создание названия 
    с автором и датой
\end{itemize}

\begin{block}{Пример}
    \begin{lstlisting}
    \documentclass{article}
    
    \title{My first document in \LaTeX}
    \author{Me}
    \date{\today}
    
    \begin{document}
        \maketitle
        Hello, World!
    \end{document}
    \end{lstlisting}
\end{block}

\end{frame}
